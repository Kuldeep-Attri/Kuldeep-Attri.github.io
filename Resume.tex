%-------------------------
% Resume in Latex
% Author : Kuldeep Sharma
%------------------------

\documentclass[letterpaper,11pt]{article}

\usepackage{latexsym}
\usepackage[empty]{fullpage}
\usepackage{titlesec}
\usepackage{marvosym}
\usepackage[usenames,dvipsnames]{color}
\usepackage{verbatim}
\usepackage{enumitem}
\usepackage[hidelinks]{hyperref}
\usepackage{fancyhdr}
\usepackage[english]{babel}
\usepackage{tabularx}
\input{glyphtounicode}


%----------FONT OPTIONS----------
% sans-serif
% \usepackage[sfdefault]{FiraSans}
% \usepackage[sfdefault]{roboto}
% \usepackage[sfdefault]{noto-sans}
% \usepackage[default]{sourcesanspro}

% serif
% \usepackage{CormorantGaramond}
% \usepackage{charter}


\pagestyle{fancy}
\fancyhf{} % clear all header and footer fields
\fancyfoot{}
\renewcommand{\headrulewidth}{0pt}
\renewcommand{\footrulewidth}{0pt}

% Adjust margins
\addtolength{\oddsidemargin}{-0.5in}
\addtolength{\evensidemargin}{-0.5in}
\addtolength{\textwidth}{1in}
\addtolength{\topmargin}{-.5in}
\addtolength{\textheight}{1.0in}

\urlstyle{same}

\raggedbottom
\raggedright
\setlength{\tabcolsep}{0in}

% Sections formatting
\titleformat{\section}{
  \vspace{-4pt}\scshape\raggedright\large
}{}{0em}{}[\color{black}\titlerule \vspace{-5pt}]

% Ensure that generate pdf is machine readable/ATS parsable
\pdfgentounicode=1

%-------------------------
% Custom commands
\newcommand{\resumeItem}[1]{
  \item\small{
    {#1 \vspace{-2pt}}
  }
}

\newcommand{\resumeSubheading}[4]{
  \vspace{-2pt}\item
    \begin{tabular*}{0.97\textwidth}[t]{l@{\extracolsep{\fill}}r}
      \textbf{#1} & #2 \\
      \textit{\small#3} & \textit{\small #4} \\
    \end{tabular*}\vspace{-7pt}
}

\newcommand{\resumeSubSubheading}[2]{
    \item
    \begin{tabular*}{0.97\textwidth}{l@{\extracolsep{\fill}}r}
      \textit{\small#1} & \textit{\small #2} \\
    \end{tabular*}\vspace{-7pt}
}

\newcommand{\resumeProjectHeading}[2]{
    \item
    \begin{tabular*}{0.97\textwidth}{l@{\extracolsep{\fill}}r}
      \small#1 & #2 \\
    \end{tabular*}\vspace{-7pt}
}

\newcommand{\resumeSubItem}[1]{\resumeItem{#1}\vspace{-4pt}}

\renewcommand\labelitemii{$\vcenter{\hbox{\tiny$\bullet$}}$}

\newcommand{\resumeSubHeadingListStart}{\begin{itemize}[leftmargin=0.15in, label={}]}
\newcommand{\resumeSubHeadingListEnd}{\end{itemize}}
\newcommand{\resumeItemListStart}{\begin{itemize}}
\newcommand{\resumeItemListEnd}{\end{itemize}\vspace{-5pt}}

%-------------------------------------------
%%%%%%  RESUME STARTS HERE  %%%%%%%%%%%%%%%%%%%%%%%%%%%%


\begin{document}

%----------HEADING----------
% \begin{tabular*}{\textwidth}{l@{\extracolsep{\fill}}r}
%   \textbf{\href{http://sourabhbajaj.com/}{\Large Sourabh Bajaj}} & Email : \href{mailto:sourabh@sourabhbajaj.com}{sourabh@sourabhbajaj.com}\\
%   \href{http://sourabhbajaj.com/}{http://www.sourabhbajaj.com} & Mobile : +1-123-456-7890 \\
% \end{tabular*}

\begin{center}
    \textbf{\Huge \scshape Kuldeep Sharma} \\ \vspace{1pt}
    \small +81 070-4005-1636 $|$ \href{https://kuldeep-attri.github.io}{\underline{kuldeep-attri.github.io}} $|$ 
    \href{https://linkedin.com/in/kuldeepattri}{\underline{linkedin.com/in/kuldeepattri}} $|$
    \href{https://github.com/Kuldeep-Attri}{\underline{github.com/kuldeep-attri}}
\end{center}


%-----------EXPERIENCE-----------
\section{Experience}
  \resumeSubHeadingListStart

    \resumeSubheading
      {Pactera APAC}{Jan. 2021 -- Present}
      {Senior AI Consultant | ‘Python, PyTorch, OpenCV, MeCab, CNNs, AWS, Git’}{Tokyo, Japan}
      \resumeItemListStart
        \resumeItem{Building probabilistic \emph{WordCorrector} model for Japanese language, used \emph{Sent2Word Tockenizer} \& \emph{Bayes Theorem}}
        \resumeItem{Designing a RPA framework for digitization of \emph{Japanese Paper Documents} by analysing both content \& structure}
        \resumeItem{Leading a team to develop an \emph{AI Receptionist} system, customized \emph{facenet-pytorch} for \emph{Face Recognition} \& \emph{Tracking}}
      \resumeItemListEnd
% -----------Multiple Positions Heading-----------
%    \resumeSubSubheading
%     {Software Engineer I}{Oct 2014 - Sep 2016}
%     \resumeItemListStart
%        \resumeItem{Apache Beam}
%          {Apache Beam is a unified model for defining both batch and streaming data-parallel processing pipelines}
%     \resumeItemListEnd
%    \resumeSubHeadingListEnd
%-------------------------------------------

    \resumeSubheading
      {AWL Inc.}{Dec. 2019 -- Dec. 2020}
      {AI Researcher | `Python, PyTorch, TensorFlow, OpenVINO, OpenCV, CNNs, Git'}{Tokyo, Japan}
      
    \resumeItemListStart
        \textbf{\resumeItem{State Estimation \& Analysis for Retail Industries}}
        \resumeItemListStart
            \resumeItem{Built a \emph{State Estimation} model using \emph{Multi-Label Cls.}, improved accuracy by \emph{6-8\%} using temporal consistency}
            \resumeItem{Designed an architecture to learn correlation in labels using \emph{Attention Mechanism}, improved F1-score by \emph{3-5}\%}
            \resumeItem{Successfully deployed \emph{Quantized} models on an edge-device with \emph{TRL-9} and realized operation in over \emph{5} stores}
        \resumeItemListEnd
        \vspace{5pt}
        \textbf{\resumeItem{AWL Trainer: Better Model Training Pipeline}}
        \resumeItemListStart
            \resumeItem{Designed a pipeline that auto-trains models for adapting to variations in data thus reducing man-hour \& cost}
            \resumeItem{Utilized generator models and fine-tuned GANs e.g. \emph{StyleGAN} \& \emph{UGATIT} for synthetic datasets generation}
            \resumeItem{Implemented a \emph{Contrastive Self-Supervised Learning} for \emph{Domain Adaptation} to auto-train domain differences}
        \resumeItemListEnd
    \resumeItemListEnd
     \vspace{2pt}
      
      \resumeSubheading
      {TUMCREATE \emph{(TUM} \&\emph{ NTU Collaboration)}}{Nov. 2017 -- Nov. 2019}
      {ML Researcher | `Python, C++, PyTorch, Caffe, OpenCV, CNNs, Git'}{NTU, Singapore}
      \resumeItemListStart
        \resumeItem{Developed \& deployed \emph{10+} real-time vision based IoT devices for Intelligent Traffic Monitoring in NTU campus}
        \resumeItem{Designed a novel CNN pruning technique, achieved \emph{10x speed up} \& \emph{7x size reduction} of CNNs without losing acc.}
        \resumeItem{Implemented using Caffe's C++ API on CPUs, presented working prototypes to {\emph{Ministry of Transport Singapore}}}
        
     \resumeItemListEnd
        
    \resumeSubheading
    {Scantist}{May 2019 -- Nov. 2019}
    {Data Scientist, Part-Time | `Python, MongoDB, PostgreSQL, Neo4j, Airflow, Git, Java'}{Singapore}
    \resumeItemListStart
        \resumeItem{Implemented a knowledge graph for \emph{3.8 Million Java library files} to analyse and remove any \emph{security vulnerability}}
        \resumeItem{Crawled raw library metadata to NoSQL database, processed dependencies to structure and store in PostgreSQL}
        \resumeItem{Implemented dynamic pipeline to schedule \& track upcoming raw libraries and updating them in \emph{knowledge graph}}
    \resumeItemListEnd
        
        
     

  \resumeSubHeadingListEnd


%-----------PROJECTS-----------
\section{Projects}
    \resumeSubHeadingListStart
      \resumeProjectHeading
          {\textbf{Scene Understanding for Autonomous Robots} $|$ \emph{`Python, Caffe, OpenCV, C++, Git'}}{May 2016 -- Dec. 2016}
          \resumeItemListStart
            \resumeItem{Developed a \emph{Scene Understanding} visual system for robots, worked on \emph{Objects Classification, Detection} \& \emph{Tracking}}
            \resumeItem{Improved system robustness(\emph{10-15\%}) by incorporating \emph{KLT Feature Tracker} \&\emph{ Kalman Filter} with \emph{ResNet-101}}
            \resumeItem{Implemented programs for \emph{6D pose estimation} of objects with \emph{Camera Calibration}, used them for localizing objects}
          \resumeItemListEnd
      \resumeProjectHeading
          {\textbf{Mesh Generation of Human Models} $|$ \emph{`C++, MATLAB, Git'}}{May 2015 -- Apr. 2016}
          \resumeItemListStart
            \resumeItem{Implemented a \emph{3D Mesh Generation} framework for solid object given nodes location using \emph{Delaunay Triangulation}}
            \resumeItem{Improved for \emph{Human Body Mesh Generation}, merged with a simulation software to study impacts during accidents}
          \resumeItemListEnd
    \resumeSubHeadingListEnd



%-----------PUBLICATION-----------
\section{Publication}
    \resumeSubHeadingListStart
      \resumeProjectHeading
          {\textbf{Evaluating the Merits of Ranking in Structured Network Pruning}}{ICDSC EAI 2020, Singapore}
          \resumeItemListStart
            \resumeItem{Studied plastic behavior of CNNs, contradicted a common belief and presented a simple random channel pruning}
            \resumeItem{Proposed a novel \& simple GFLOPs-aware iterative CNN pruning technique, can lower the inference time by \emph{15\%} }
          \resumeItemListEnd
    \resumeSubHeadingListEnd



%-----------EDUCATION-----------
\section{Education}
  \resumeSubHeadingListStart
    \resumeSubheading
      {Indian Institute of Technology Delhi}{New Delhi, India (2013 -- 17)}
      {Bachelor of Technology in Industrial Engineering}{\textbf{Top 0.1\% in JEE 2013}} \\ 
      \vspace{9pt}
      \textbf{\underline{Major Focus}}{: \emph{Machine Learning, Computer Vision, Attention Mechanism, Object Recognition} \& \emph{Detection, Linear Algebra, Data Structure} \& \emph{Algorithms, Probability} \& \emph{Statistics, Graph Algorithms, OS}}
  \resumeSubHeadingListEnd



%
%-----------PROGRAMMING SKILLS-----------
\section{Technical Skills \& Interests}
 \begin{itemize}[leftmargin=0.15in, label={}]
    \small{\item{
     \textbf{Languages \& Tools}{: Python(\textit{PyTorch, TensorFlow, Caffe, OpenCV}), C++, Swift(iOS), MATLAB, R, AWS, Linux, CUDA, Git, Docker, MongoDB, PostgreSQL, Airflow, Neo4j, JIRA, Trello, Google Colab, VS Code, Microsoft Office}} \\
     \textbf{Interests}{: Travelling, Pensive Discussions, Running, Cycling, History, Astronomy,  DOTA2 \& Reading Research Papers}
     }
 \end{itemize}


%-------------------------------------------
\end{document}
